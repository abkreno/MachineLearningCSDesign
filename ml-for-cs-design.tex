% This is LLNCS.DEM the demonstration file of
% the LaTeX macro package from Springer-Verlag
% for Lecture Notes in Computer Science,
% version 2.4 for LaTeX2e as of 16. April 2010
%
\documentclass{llncs}
%
\usepackage{makeidx}  % allows for indexgeneration
%
\begin{document}
%
\frontmatter          % for the preliminaries
%
\pagestyle{headings}  % switches on printing of running heads
\addtocmark{Machine Learning for Constraint Solver design} % additional mark in the TOC
%
\title{Machine Learning for Constraint Solver design (Alldiff Constraint)}
%
\titlerunning{Machine Learning for Constraint Solver design}  % abbreviated title (for running head)
%                                     also used for the TOC unless
%                                     \toctitle is used
%
\author{Mohamed Mostafa El Hamamsy}
\institute{German University in Cairo, New Cairo City \\ Main Entrance of Al Tagamoa Al Khames, Egypt\\
\email{mohamed.el-hamamsy@student.guc.edu.eg}}

\maketitle              % typeset the title of the contribution

\begin{abstract}
A Constraint Satisfaction Problem (CSP) with the All Different constraint forces every variable in the given satisfaction group to be different from the other variables. This problem is solvable using many different techniques, some of them are naive and others are more sophisticated techniques such as, the Generalized Arc Consistency (GAC) technique. Selecting an algorithm to solve the alldiff CSP is a hard decision to make. In this paper we discuss the usage of Algorithms Selection using Machine Learning to solve the alldiff CSP.

\keywords{All Different, Generalized Arc Consistency, Constraints Solvers, Algorithms Selection}

\end{abstract}
%
\section{Introduction}
%
% \cite{ml:csd}
Constraints programming is a technology that has been proven successful for solving many complex combinatorial decision or optimization problems, such as; scheduling, industrial design, aviation and banking, to name but a few examples \cite{ml:csd}.
\\
\\
One of the most known problems addressed by Constraints programming is the Constraints Satisfaction Problem (CSP). When designing a solver for a CSP and modeling a CSP problem, there are many design decisions that has to be made in order to reach a solver with good performance. These decisions can be such as; the level of consistency to use and what data structures shall be used to allow the solver to backtrack \cite{ml:csd}. Usually these design decisions are done manually by a human being and this increases the possibility for errors due to lack of experience or other factors. Also, once a decision has been made it's going to be static. In other words, even if a decision with a better performance can be applied to a given problem instance, it's not going to be easy to change that decision and so, the performance will not always be optimal \cite{ml:csd}.

In this paper we discuss a solution to this problem by looking from a Machine Learning perspective. The idea here is that, we train a Classifier that given a problem class or a problem instance shall be able to decide automatically which design decision should be made to solve the problem instance. With this approach the problem of depending on human choices is solved. In other words we no longer need to depend on manual decisions based on human experience, and even after the design decision has been taken for a problem instance it's no longer going to be static and will be changed optimally for a given problem class.
\\
\\
We are going to describe how the Classifier will work starting from modeling the problem as a training instance to the classifier, passing by how the training will actually happen and finally how to use the classifier and what was the reached results.

%
\section{Background}
%
Formally speaking a CSP is defined as a set of variables: $X_{1}, X_{2},...,X_{n}$, each variable has its own domain: $D_{1}, D_{2},...,D_{n}$ and there are some constraints on these variables: $C_{1}, C_{2},...,C_{m}$, each constraint $C_{i}$ consists of a subset of variables along with the allowable assignment values for each of these variables. A solution to a CSP is an assignment that assigns a value to each variable $X_{i}$ from its domain $D_{i}$, and this assignment must satisfy all the constraints specified to the problem instance.
\\
\\
In this paper we address solving one of the known Constraints which is the All-different (alldiff) Constraint. A CSP with the alldiff Constraint has a set of rules that forbids the equality on a set of variables. In other words given a CSP problem with a set of variables $X_{1},...,X_{n}$  where for each variable there is a finite domain $D_{1},...,D_{m}$. Then:
\begin{eqnarray*}
  \textrm{alldiff}(X_{1},...,X_{n}) = \{(v_{1},...,v_{n}) | v_{i} \in D_{i}, v_{i} \neq v_{j} \textrm{ for } i \neq j\}
\end{eqnarray*}
\\
\\

In order to solve a CSP with the alldiff constraint there are several solvers that can be applied. But before getting into this lets first discuss the most naive approach. To illustrate how it works here is an example:
\\\\say we have 3 variables:
\begin{eqnarray*}
  X_{1}, X_{2}, X_{3}
\end{eqnarray*}
and we have the constraint alldiff on these variables:
\begin{eqnarray*}
  \textrm{alldiff}(X_{1}, X_{2}, X_{3})
\end{eqnarray*}
the naive approach will decompose the alldiff constraint into the following set of constraints:
\begin{eqnarray*}
  X_{1} \neq X_{2}\\
  X_{1} \neq X_{3}\\
  X_{2} \neq X_{3}
\end{eqnarray*}
Finding a solution to the model in the previous example can be done by systematically enumerating all the possible values combinations. Then, for each combination we check whether it satisfies all the constraints or not, if it does then we have found a solution, otherwise the search continue. Unfortunately, the naive approach is too slow, the enumeration of all the possible combinations takes a lot of time. In fact it leads to a search space of exponential size. Thats why one of the main problems is that it's not very efficient, for example, consider the case where we have:
\begin{eqnarray*}
  X_{1}, X_{2}, X_{3}, X_{4}
\end{eqnarray*}
and the domain of these 3 variables is the same:
\begin{eqnarray*}
  D_{1} = D_{2} = D_{3} = D_{4} = \{1, 2, 3\}
\end{eqnarray*}
and we have the constraint:
\begin{eqnarray*}
  \textrm{alldiff}(X_{1}, X_{2}, X_{3}, X_{4})
\end{eqnarray*}
It's easy to see that there will never be an assignment to the variables from these domains that will satisfy the alldiff constraint, However, this knowledge cannot be derived when just considering the decomposition into pairs of variables \cite{ml:csd}. So this approach is non-practical in this case and specially when the number of constraints is large.
\\\\
There are several other approaches to solve the alldiff constraints. One of them is the Generalized Arc Consistency (GAC) approach. The GAC is a more sophisticated approach that mainly tries to simplify the CSP instance in a way that will make it easier to solve. The general idea about the GAC is that it tries to optimize the domains by doing more propagation. Refer to \cite{gac:alldiff} for more details about the GAC alldiff algorithm.
\\\\
Algorithm Selection is a meta-algorithmic technique to dynamically choose an algorithm from a set that will solve a problem instance in the best performance. This is based on the idea that, when using a static algorithm it can perform well on a subset of problem instances, But there will also be another subset on which the algorithm will perform poorly, while other algorithms can perform better on that subset. So what algorithm selection aims to do is to make the best algorithm choice every time a problem instance is applied.
\\\\
In Machine Learning Algorithm Selection is known as \textit{meta-learning}, what we do in this paper is discuss the training of a classifier that for a given problem instance aims to decide the most efficient algorithm to solve the instance. By training the classifier on a prepared set of problem instances and then using the trained meta-classifier for classification.  

%
% ---- Bibliography ----
%
\begin{thebibliography}{5}
%

\bibitem {ml:csd}
Ian P. Gent, Lars Kotthoff, Ian Miguel, Peter Nightingale:
Machine learning for constraint solver design -- {A} case study for the alldifferent constraint,
CoRR, 1008.4326 (2010)

\bibitem {gac:alldiff}
Gent, Ian P. and Miguel, Ian and Nightingale, Peter:
Generalised Arc Consistency for the AllDifferent Constraint: An Empirical Survey,
Artif. Intell. , 1973--2000, (2008)

\end{thebibliography}
% \clearpage
% \addtocmark[2]{Author Index} % additional numbered TOC entry
% \renewcommand{\indexname}{Author Index}
% \printindex
% \clearpage
% \addtocmark[2]{Subject Index} % additional numbered TOC entry
% \markboth{Subject Index}{Subject Index}
% \renewcommand{\indexname}{Subject Index}
% \input{subjidx.ind}
\end{document}
